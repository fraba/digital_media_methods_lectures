\documentclass[serif, aspectratio=169]{beamer}

\usetheme{metropolis}

% Define title
\newcommand{\doctitle}{Digital Research Methods II}
\newcommand{\docsubtitle}{Subtitle}% Remove if not needed

% Define author
\newcommand{\docauthor}{Francesco Bailo}
% \newcommand{\docauthortitle}{PhD Student}
\newcommand{\docauthorinstitute}{The University of Sydney}
\newcommand{\docauthoremail}{francesco.bailo@sydney.edu.au}
\newcommand{\InstitutePlace}{Institue and place}

% Define subject and keywords
\newcommand{\docsubject}{ADD SUBJECT HERE}
\newcommand{\dockeywords}{Keywords;}

% Define creation date (format D:YYYYMMDDHHmmss)
\newcommand{\doccreationdate}{D:201411081900}

\input{/Users/francesco/tex_inputs/biblatex_apa}
\input{/Users/francesco/tex_inputs/default_slides_set}


% Beamer buttons

\setbeamertemplate{button}{\tikz
  \node[
  inner xsep=4pt,
  draw=structure!80,
  fill=structure!50,
  rounded corners=4pt]  {\tiny\insertbuttontext};}

%----------------------------------------------------------------------------------------
%	TITLE SECTION
%----------------------------------------------------------------------------------------

\title{\doctitle} % Article title

\author{
\large
\textsc{Francesco Bailo}\\[2mm] 
\normalsize The University of Sydney \\
% \normalsize \href{mailto:\docauthoremail}{\docauthoremail} 
% \vspace{-5mm}
}

\titlegraphic{\includegraphics[height=0.5cm]{/Users/francesco/Dropbox/Thesis_PhD/figure/Usyd_logo.png}}

\date{3 May 2018}

\usepackage[titletoc]{appendix}

%----------------------------------------------------------------------------------------


%----------------------------------------------------------------------------------------
%	HEADER AND FOOTER SECTION
%----------------------------------------------------------------------------------------

\setbeamertemplate{headline}{%
\leavevmode%
  \hbox{%
    \begin{beamercolorbox}[wd=\paperwidth,ht=2.5ex,dp=1.125ex]{palette quaternary}%
    \insertsectionnavigationhorizontal{\paperwidth}{}{\hskip0pt plus1filll}
    \end{beamercolorbox}%
  }
}


\setbeamertemplate{footline}[text line]{%
  \parbox{\linewidth}{\vspace*{-8pt}\href{mailto:\docauthoremail}{\docauthoremail}\hfill\insertpagenumber/\insertpresentationendpage
}}

%----------------------------------------------------------------------------------------

\newcommand{\customcite}[1]{\citeauthor{#1}, \citetitle{#1}, \citeyear{#1}}


\usepackage{newverbs}
\newverbcommand{\bverb}
  {\begin{lrbox}{\verbbox}}
  {\end{lrbox}\colorbox{gray!30}{\box\verbbox}}

\begin{document}

{
\setbeamertemplate{headline}{} 
\setbeamertemplate{footline}{} 
\begin{frame}
  \titlepage
\end{frame}
}
\addtocounter{framenumber}{-1}

\frame{\tableofcontents}


\section{Observing behaviour}

\begin{frame}

\begin{columns}
\begin{column}{0.4\textwidth}

This section is largely based on Chapter 2 of \citetitle{salganik_bit_2018} by Matthew Salganik (2018).

\end{column}

\begin{column}{0.6\textwidth}

\begin{figure}

\includegraphics[width=0.7\textwidth]{figure/9780691158648.png}

\end{figure}

\end{column}
\end{columns}

\end{frame}

\begin{frame}

Whatever the subject of your research, there are mainly three ways to collect data: 

\begin{enumerate}

\item<1-> Running experiments

\item<2-> Asking questions

\item<3-> Observing behaviour $\leftarrow$

\end{enumerate}

\begin{itemize}
\item <4-> Observational data are collected without interfering with either
\begin{itemize}
\item<4-> the subject of the investigation or 
\item<4-> the environment of the subject of the investigation.
\end{itemize}
\end{itemize}

\end{frame}

\begin{frame}

\begin{enumerate}

\item Navigate to {\url{https://socrative.com}}
\item \enquote{Student login}
\item Room name: \enquote{BAILO}

\end{enumerate}

\end{frame}

\begin{frame}

\begin{columns}
\begin{column}{0.4\textwidth}

Observation of something or somebody is the primordial way of investigating what we are interested in (and usually the beginning of an investigation). \\~\

The use of instruments and sensors to observe and record what we observe is not new.  \\~\

What is new is the number of instruments and sensors monitoring and recording human behaviour.


\end{column}
\begin{column}{0.6\textwidth}

\begin{figure}

\includegraphics[width=0.7\textwidth]{figure/Bertini_fresco_of_Galileo_Galilei_and_Doge_of_Venice}
\caption{\footnotesize{\citetitle{bertini_galileo_1858}, \cite{bertini_galileo_1858}}}
\end{figure}

\end{column}
\end{columns}

\end{frame}


\begin{frame}

The combination of \textit{flow} and the \textit{stock} of data produced by these instruments and sensors is often called \textbf{Big Data}. 

Big data are \textit{big} on three dimensions: 

\begin{itemize}

\item \textbf{V}olume
\item \textbf{V}ariety
\item \textbf{V}elocity

\end{itemize}

\end{frame}

\begin{frame}

\begin{itemize}

\item<1> So... can you think of any example of big data or source of big data? 

\end{itemize}

\begin{itemize}

\item<2-> Big data are not only data generated by the online activity of users and are not only created by companies. 

\end{itemize}

\begin{enumerate}

\item<3-> Big data are generated online but also offline every time a sensor records a human behaviour. 

\item<4-> Big data are created by companies but also by governments.

\end{enumerate}

\end{frame}


\begin{frame}

\enquote{Big data are created and collected by \textbf{companies} and \textbf{governments} for purposes other than research. Using this data for research therefore requires repurposing.} \autocite[14]{salganik_bit_2018}\\~\ \\~\
\begin{columns}
\begin{column}{0.4\textwidth}
\centering\includegraphics[width=0.5\textwidth]{figure/300px-Twitter_bird_logo_2012}
\end{column}

\begin{column}{0.2\textwidth}
\LARGE\centering$\neq$
\end{column}

\begin{column}{0.4\textwidth}
\centering\includegraphics[width=0.7\textwidth]{figure/640px-ABS_Census_Logo}
\end{column}

\end{columns}

\end{frame}

\begin{frame}

According to \autocite{salganik_bit_2018}, Big Data share 10 characteristics. 

\begin{enumerate}

\item<1-> Big Data are \textbf{big}: Rare events, heterogeneity, small differences. \textit{But} how data were created?

\item<2-> Big Data are \textbf{always-on}: Unexpected events and real-time estimates. \textit{But} the systems that collected the data are constantly changing (see \textit{drifting} later)!

\item<3-> Big Data are \textbf{nonreactive}: Measurement is less likely to change behaviour. \textit{But} a social desirability bias persist. 

\item<4-> Big Data are \textbf{incomplete}: No demographic information, no information on behaviour on other platforms, and no data to operationalise theoretical constructs (e.g. \enquote{intelligence}).

\item<5-> Big Data are \textbf{inaccessible}: Access is controlled and conditional. 

\end{enumerate}

\end{frame}

\begin{frame} 

\begin{enumerate}

\setcounter{enumi}{5}

\item<1-> Big Data are \textbf{non-representative}: Data do not come from a probabilistic random sample of the population. 

\item<2-> Big Data are \textbf{drifting}: Population drift, behavioural drift, system drift. Systems keep changing all the time!

\item<3-> Big Data are \textbf{algorithmically confounded}: Engineering choices impact user behaviours. Also, performativity issues.

\item<4-> Big Data are \textbf{dirty}: Dirty data can be created unintentionally or intentionally (e.g. bots). 

 \item<5-> Big Data are \textbf{sensitive}: The potential sensitivity of the data is difficult to always assess. 

\end{enumerate}

\end{frame}

\section{Network analysis}

\subsection{A very short introduction}

\begin{frame}

\begin{block}{Relations, not attributes. Networks, not groups.}
[S]ocial network analysts argue that causation is not located in the individual, but in the social structure. While people with similar attributes may behave similarly, explaining these similarities by pointing to common attributes misses the reality that \textit{individuals with common attributes often occupy similar positions in the social structure}. That is, \textit{people with similar attributes frequently have similar social network positions}. Their similar outcomes are caused by the \textbf{constraints}, \textbf{opportunities} and \textbf{perceptions} created by these similar network positions. \autocite[13]{marin_social_2011}
\end{block}

\end{frame}

\begin{frame}

\begin{columns}
\begin{column}{0.6\textwidth}
\begin{figure}
    \centering
    \includegraphics{figure/undirected_network}
\caption{Traditional visualisation of two small networks...}
\end{figure}
\end{column}
\begin{column}{0.4\textwidth}
\begin{figure}
    \centering
    \includegraphics{figure/undirected_adj_matrix}
\caption{... and the adjacency matrix of the left-hand network \autocite[111]{newman_networks_2010}.}
\end{figure}
\end{column}
\end{columns}
\end{frame}

\begin{frame}

\begin{columns}
\begin{column}{0.5\textwidth}
\begin{figure}
    \centering
    \includegraphics[width = 0.7\textwidth]{figure/directed_network}
\caption{A directed network...}
\end{figure}
\end{column}
\begin{column}{0.5\textwidth}
\begin{figure}
    \centering
    \includegraphics{figure/directed_adj_matrix}
\caption{... and its adjacency matrix (not symmetric!) \autocite[112]{newman_networks_2010}.}
\end{figure}
\end{column}
\end{columns}
\end{frame}

\begin{frame}
{Network measures}

\begin{columns}
\begin{column}{0.6\textwidth}
  \begin{description}
  \item<1-> [Degree of a vertex] number of connections
  \item<2-> [Authority of a vertex] number of important connections
  \item<3-> [Closeness of a vertex] mean distance to other vertices
  \item<4-> [Betweenness of a vertex] extent to which a vertex lies on paths between other vertices
  \item<5-> [Group of vertices]
  \end{description}
\end{column}
\begin{column}{0.4\textwidth}
\begin{figure}
    \includegraphics[width=0.55\linewidth]<4->{figure/betweenness}
\end{figure}
\begin{figure}\vspace{.5 cm}
    \includegraphics[width=0.55\linewidth]<5->{figure/groups}
\end{figure}
\begin{figure}
    \includegraphics[width=0.55\linewidth]<6->{figure/components}
\end{figure}
\end{column}
\end{columns}

\end{frame}


\begin{frame}
{Network measures}

\begin{columns}
\begin{column}{0.6\textwidth}
  \begin{description}
  \item<1-> [Transitivity of edges] Alice \textit{friend of} Bob \textit{friend of} Cat \textit{friend of} Alice
\item<2-> [Reciprocity of edges] Alice  \textit{friend of} Bob \textit{friend of} Alice
\item<3-> [Similarity of vertices] extent to which the \textit{neighbourhood} of vertices is similar
\item<4-> [Homophily of vertices] tendency to associate with similar vertices
  \end{description}
\end{column}
\begin{column}{0.4\textwidth}
\begin{figure}
    \includegraphics[width=0.35\linewidth]<1->{figure/transitivity}
\end{figure}
\begin{figure}
    \includegraphics[width=0.4\linewidth]<2->{figure/reciprocity}
\end{figure}
\begin{figure}
    \includegraphics[width=0.4\linewidth]<3->{figure/similarity}
\end{figure}
\end{column}
\end{columns}

\end{frame}

\begin{frame}

\begin{figure}
    \centering
    \includegraphics[width=0.55\textwidth]{figure/race_network}
\caption{Friendship network at a US high school \autocite[221]{newman_networks_2010}.}
\end{figure}

\end{frame}

\begin{frame}
{Community detection}

The goal of a community detection algorithm is simply to separate nodes into groups that have only a few edges \textit{between} them and many edges \textit{within}.

\begin{figure}
    \includegraphics[width=0.4\linewidth]{figure/sample_smallworld}
\caption{A randomly generated network with 100 vertices and 300 edges}
\end{figure}

\end{frame}

\begin{frame}
{Community detection}

How many communities do you see in this network? (Go to: socrative.com, room: BAILO)

\begin{figure}
    \includegraphics[width=0.4\linewidth]{figure/sample_smallworld}
\end{figure}

\end{frame}

\begin{frame}
{Community detection}

\begin{figure} \includegraphics{figure/sample_smallworld_communities}
\end{figure}

\end{frame}


\begin{frame}

\begin{columns}

\begin{column}{0.6\textwidth}
\begin{figure}
    \includegraphics{figure/paper_bailo}
\end{figure}
\end{column}

\begin{column}{0.4\textwidth}
\begin{figure}
    \includegraphics{figure/paper_network}
\end{figure}
\end{column}

\end{columns}

\end{frame}

\subsection{Tools}

\begin{frame}

\centering \textbf{Easy, small n}: Gephi (\url{gephi.org})

\begin{figure}
    \centering
    \includegraphics[width=0.65\textwidth]{figure/gephi.jpg}
\end{figure}

\end{frame}

\begin{frame}

\centering \textbf{Hard, big n} igraph package (\url{igraph.org}) in R (\url{www.r-project.org}) or Python (\url{www.python.org})

\begin{figure}
    \centering
    \includegraphics[width=0.6\textwidth]{figure/igraph}
\end{figure}

\end{frame}

\subsection{Resources}

\begin{frame}

Getting started bibliography:

\begin{description}
\item [Easy] \customcite{scott_what_2012}
\item [Important] \customcite{marin_social_2011}
\item [Hard] \customcite{newman_networks_2010}
\end{description}

\end{frame}


\begin{frame}

Tutorials for beginners by Katherine Ognyanova (Rutgers University):

\begin{figure}
    \centering
    \includegraphics[width=0.1\textwidth]{figure/Ognyanova_Katherine}
\end{figure}


\begin{itemize}
\item Network visualisation with Gephi (\url{kateto.net/sunbelt2016})
\item Network visualization with R (\url{kateto.net/network-visualization})
\item Network Analysis and Visualization with R and igraph (\url{kateto.net/networks-r-igraph})
\end{itemize}

\end{frame}

\section{Text analysis}

\subsection{Another very short introduction}

\begin{frame}

Quantitative text analysis is necessary when the manual coding of documents in not feasible or acceptable. 

When you face a large \textbf{corpus} of \textbf{documents}, you might want some methods to automatically:

\begin{enumerate}

\item Find patterns within the documents,

\item Compare (and maybe group) documents.

\end{enumerate}

\end{frame}

\begin{frame}[fragile]
\frametitle{Finding patterns}

A textual pattern is as simple as \texttt{dog}.

\begin{itemize}

\item Finding patterns doesn't involve any statistical analysis.

\item But you might need to use of regular expressions (a.k.a. \enquote{regex}) if your pattern is complex.

\end{itemize}

\end{frame}

\begin{frame}[fragile]
\frametitle{Finding patterns}

Let's say, that you want to find in your corpus all the instances of \texttt{dog} and \texttt{cat}.

\begin{description}

\item<1->[You want to find] \enquote{I have two \underline{dog}s and a \underline{cat}} or \enquote{\underline{Cat}s are felines}

\item<2->[But you don't want to find]  \enquote{the \underline{cat}egorization of syntactic \underline{cat}egories}

\item<3-> You need a regular expression like: {\LARGE\verb=\b(cats?|dogs?)\b=} 

\end{description}

{\tiny (\href{https://regexr.com/3oqld}{link to interactive example})}

\end{frame}

\begin{frame}[fragile]
\frametitle{Finding patterns}

A few simple regex topics:

\begin{description}

\item<1->[Quantifier] \bverb|?|

\begin{itemize}

\item<2-> \bverb|abc?| matches a string that has \enquote{ab} followed by zero or one \enquote{c}

\end{itemize}

\item<3->[OR operator] \bverb=|=

\begin{itemize}

\item<4-> \bverb=a(b|c)= matches a string that has \enquote{a} followed by \enquote{b} or \enquote{c}

\end{itemize}

\item<5->[Boundaries]\bverb|\b|

\begin{itemize}

\item<6-> \bverb=\babc\b= matches only a whole word

\end{itemize}

\end{description}

{\footnotesize Exercise: Go to \texttt{\href{https://regexr.com/3os9b}{regexr.com/3os9b}} (not with Explorer) and enter a regular expression to match \enquote{France} but also \enquote{French}}.

\end{frame}

\begin{frame}
\frametitle{Comparing documents}

Comparing documents involves statistical analysis and matrix algebra (while finding patterns doesn't). It usually relies on Natural-language processing (NLP), the branch of computer science that studies the human language and its interactions with the machines. 

In its most primordial application, NLP treats documents as \textbf{bag-of-words}:
\begin{itemize}
\item  The \textit{position} of terms within the document is disregarded,
\item What counts is the \textit{frequency} of the terms.
\end{itemize}

\end{frame}

\begin{frame}
\frametitle{Comparing documents}

Let's see how we process documents in a common NLP application.

\begin{itemize}

\item<1-> We remove from the documents all the stop-words;

\item<2-> doc1 = "drugs hospitals doctors" \\
doc2 = "smog pollution environment" \\
doc3 = "doctors hospitals healthcare" \\
doc4 = "pollution environment water" \\

\item<3-> We count the frequency of each term in each document an we produce a term-document matrix

\end{itemize}

\end{frame}

\begin{frame}
\frametitle{Comparing documents}

\begin{table}[ht]
\centering
\begin{tabular}{rrrrr}
  \hline
 & doc1 & doc2 & doc3 & doc4 \\ 
  \hline
doctor & 1 & 0 & 1 & 0 \\ 
  drug & 1 & 0 & 0 & 0 \\ 
  environ & 0 & 1 & 0 & 1 \\ 
  healthcar & 0 & 0 & 1 & 0 \\ 
  hospit & 1 & 0 & 1 & 0 \\ 
  pollut & 0 & 1 & 0 & 1 \\ 
  smog & 0 & 1 & 0 & 0 \\ 
  water & 0 & 0 & 0 & 1 \\ 
   \hline
\end{tabular}
\caption{Term-document matrix. Terms were stemmed.} 
\label{tab:example1-tf-table}
\end{table}

\end{frame}

\subsection{Tools}

\begin{frame}

\begin{itemize}

\item Nvivo (\url{www.qsrinternational.com/nvivo})

\item Regular Expression (\url{regexr.com})

\item R (\url{www.r-project.org}) or Python (\url{www.python.org})

\end{itemize}

\end{frame}

\subsection{Resources}

\begin{frame}

\begin{description}

\item [Introductory] \customcite{jockers_text_2014}

\item [Introductory] \customcite{bird_natural_2009}

\item [Hard] \customcite{manning_introduction_2008}

\end{description}

\end{frame}

\section{Ethics}

\begin{frame}
{Issues with relational data}

\begin{columns}

\begin{column}{0.4\textwidth}
\begin{figure}
    \includegraphics[width=1\textwidth]{figure/garcia_paper}
\caption{\customcite{sarigol_online_2014}}
\end{figure}
\end{column}

\begin{column}{0.25\textwidth}
\begin{figure}
    \includegraphics[width=1\textwidth]{figure/garcia_fig1}
\end{figure}
\end{column}

\begin{column}{0.35\textwidth}
\begin{figure}
    \includegraphics[width=1\textwidth]{figure/garcia_fig2}
\end{figure}
\end{column}
\end{columns}

\end{frame}

\begin{frame}
{Ethics in the digital age: Open issues}

\begin{itemize}

\item Public and Private space. What about online fora (e.g. Facebook public pages?)

\item Informed consent.

\item Right to privacy. But who owns the data?

\end{itemize}

\end{frame}

\nocite{salganik_bit_2018}
\nocite{bailo_hybrid_2017}


\section{Bonus: Spatial analysis}

\begin{frame}
{Spatial analysis}
\begin{columns}
\begin{column}{0.7\textwidth}
\begin{figure}
    \includegraphics[width=0.9\textwidth]{figure/john_snow_map}
\end{figure}
\end{column}
\begin{column}{0.3\textwidth}
Redrawing of John Snow's map of cases of cholera during the London outbreak of 1854 \autocite[24]{tufte_visual_2001}
\end{column}
\end{columns}
\end{frame}

\begin{frame}
{Tool for spatial analysis}

\begin{itemize}

\item QGIS (\url{qgis.org})

\end{itemize}

\begin{figure}
    \includegraphics[width=0.8\textwidth]{figure/qgis}
\end{figure}

\end{frame}


\begin{frame}[allowframebreaks]
\printbibliography
\end{frame}



\end{document}
%%% Local Variables:
%%% mode: latex
%%% TeX-master: t
%%% End:
